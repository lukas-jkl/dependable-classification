\documentclass[acmsmall,nonacm]{acmart}
\makeatletter
\renewcommand\@formatdoi[1]{\ignorespaces}
\makeatother

\usepackage{subcaption}
\usepackage{float}
\usepackage{tabularx}
\usepackage{graphicx}
\usepackage{caption}


%%
%% \BibTeX command to typeset BibTeX logo in the docs
\AtBeginDocument{%
  \providecommand\BibTeX{{%
    \normalfont B\kern-0.5em{\scshape i\kern-0.25em b}\kern-0.8em\TeX}}}


\begin{document}

%\begin{titlepage}
%\end{titlepage}

%%
%% The "title" command has an optional parameter,
%% allowing the author to define a "short title" to be used in page headers.
\title{AI Dependability Assessment}

%%
%% The "author" command and its associated commands are used to define
%% the authors and their affiliations.
%% Of note is the shared affiliation of the first two authors, and the
%% "authornote" and "authornotemark" commands
%% used to denote shared contribution to the research.

\author{Patrick Deutschmann}
\email{patrick.deutschmann@student.tugraz.at}

\author{Lukas Timpl}
\email{lukas.timpl@student.tugraz.at}


%%
%% The abstract is a short summary of the work to be presented in the
%% article.
\begin{abstract}
\end{abstract}

%%
%% The code below is generated by the tool at http://dl.acm.org/ccs.cfm.
%% Please copy and paste the code instead of the example below.
%%
%\begin{CCSXML}
%<ccs2012>
%   <concept>
%       <concept_id>10010147.10010178.10010179</concept_id>
%       <concept_desc>Computing methodologies~Natural language processing</concept_desc>
%       <concept_significance>500</concept_significance>
%       </concept>
% </ccs2012>
%\end{CCSXML}

%%
%% This command processes the author and affiliation and title
%% information and builds the first part of the formatted document.
\maketitle

\tableofcontents


\section{Introduction}

\section{Approach} 
%TODO
%• A description of the employed ML approach
%• All assumptions made, including an exhaustive proposal/justification for their validation
~\cite{xu2020automatic}

\section{Results}
%TODO
% For each dataset, an upper bound for the misclassification error, including its justification



\section{Scaling to higher dimensions}
%TODO
% An outline how the approach could be scaled to higher dimensions

\section{Appendix}
%TODO
% In an appendix, the documented code should be supplied

\pagebreak  

%%
%% The next two lines define the bibliography style to be used, and
%% the bibliography file.
\bibliographystyle{ACM-Reference-Format}

\bibliography{references}
% References to results that are used


\end{document}
\endinput

